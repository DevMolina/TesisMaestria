%\newpage
%\setcounter{page}{1}
\begin{center}
    \begin{figure}
        \centering%

        \epsfig{file=HojaTitulo/LogoUPTC,scale=0.7}%
    \end{figure}
    \thispagestyle{empty} \vspace*{0.5cm} \textbf{\huge
    Evaluación de las propiedades estructurales, magnéticas y ópticas del
    sistema
    \ce{Lu_{3}Al_{5-x}Fe_{x}O_{12}}:\ce{Ce^{3+}} obtenido por método de
    reacción de
    estado sólido}\\[4cm]
    \Large\textbf{Yeison Daniel Molina Monsalve}\\[4cm]
    \small Universidad Pedagógica y Tecnológica de Colombia\\
    Facultad de Ingeniería, Maestría en metalurgia y ciencias de los
    materiales\\
    Tunja, Colombia\\
    2021\\
\end{center}

\newpage{\pagestyle{empty}}%\cleardoublepage}

\newpage
\begin{center}
    \thispagestyle{empty} \vspace*{0cm} \textbf{\huge
    Evaluación de las propiedades estructurales, magnéticas y ópticas del
    sistema
    \ce{Lu_{3}Al_{5-x}Fe_{x}O_{12}}:\ce{Ce^{3+}} obtenido por método de
    reacción de
    estado sólido}\\[1.1cm]
    \Large\textbf{Yeison Daniel Molina Monsalve}\\[1.1cm]
    \small Trabajo de grado presentado como requisito parcial para
    optar
    al
    título de:\\
    \textbf{Magister en metalurgia y ciencias de los materiales}\\[1.1cm]
    Director\\
    Carlos Arturo Parra Vargas Ph.D.\\[0.5cm]
    Codirector:\\
    Christian Fabian Olivera Varela MSc.\\[1.1cm]
    %%L\'{\i}nea de Investigación:\\
    %%Nombrar la l\'{\i}nea de investigación en la que enmarca la tesis  o trabajo de investigación\\
    Grupo de Investigación:\\
    Grupo de Física de Materiales\\[1.1cm]
    Universidad Pedagógica y Tecnológica de Colombia\\
    Facultad de Ingeniería, Maestría en metalurgia y ciencias de los
    materiales\\
    Tunja, Colombia\\
    2022\\
\end{center}

\newpage{\pagestyle{empty}}%\cleardoublepage}

\newpage
\thispagestyle{empty} \textbf{}\normalsize
\\\\\\%
\textbf{``Siempre dicen que el tiempo cambia las cosas, pero en realidad se tienen que cambiar por uno mismo\"\\\\
Andy Warhol''}\\[4.0cm]

\begin{flushright}
    \begin{minipage}{8cm}
        \noindent
        \small
        A mi madre\\
        Por inspirarme y ser mi motivación para ser cada día mejor. \\
        [1.0cm]\\
        A mi padre\\
        Por ser una gran persona, humilde y sencilla.
        \\
        
    \end{minipage}
\end{flushright}

\newpage{\pagestyle{empty}}%\cleardoublepage}

\newpage
\thispagestyle{empty} \textbf{}\normalsize
\\\\\\%
\textbf{\LARGE Agradecimientos}
\phantomsection
\addcontentsline{toc}{chapter}{\numberline{}Agradecimientos}\\\\

Agradezco al doctor Carlos Parra por permitirme desarrollar mi trabajo de grado de maestría 
en el grupo de Investigación de Física de Materiales de la Universidad Pedagógica y Tecnológica de Colombia, 
y a su equipo de trabajo por su colaboración y asesoría en el desarrollo de esta investigación, 
especialmente a Christian Varela por su dedicación, acompañamiento y esmero.
%Esta sección es opcional, en ella el autor agradece a las personas o
%instituciones que colaboraron en la realización de la tesis  o trabajo de
%investigación. Si se incluye esta sección, deben aparecer los nombres
%completos, los cargos y su aporte al documento.\\

\newpage{\pagestyle{empty}}%\cleardoublepage}

\newpage
\textbf{\LARGE Resumen}
\phantomsection
\addcontentsline{toc}{chapter}{\numberline{}Resumen}\\\\

El granate de aluminio lutecio (LuAG), presenta una estructura prometedora como
anfitrión de diferentes tierras raras, permitiendo su aplicación en diferentes
dispositivos debido a sus propiedades ópticas. El LuAG dopado con \ce{Ce^{3+}} es
un fósforo de color amarillo, ampliamente utilizado como conversor en diodos emisores
de luz (LED). En el presente trabajo se obtuvo una nueva serie de granates 
\ce{Lu_{3}Al_{5-x}Fe_{x}O_{12}}:\ce{Ce^{3+}} ($0.0 \leq x \leq 4.5$)
mediante el método de reacción en estado sólido a 1200 \textcelsius. Los
materiales
obtenidos se caracterizaron por difracción de rayos X, refinamiento de
Rietveld, espectroscopía de reflectancia difusa UV-Vis, espectroscopía de
absorción, espectroscopía de fotoluminiscencia, microscopia electrónica de
barrido (SEM),
EDX y magnetometría de muestra vibrante (VSM). El dopaje con \ce{Fe^{3+}}
permitió
obtener materiales en fase pura a temperaturas y tiempos por debajo de los
reportados anteriormente. Por otro lado, los materiales alcanzaron una
absorción
de azul mejorada y una emisión ajustable de verde a naranja. Estas
propiedades ópticas son atribuibles a un fenómeno de desplazamiento hacia el
rojo debido a un aumento de la división del campo cristalino en los niveles de
energía de \ce{Ce^{3+}}. Además, los fósforos obtenidos exhibieron un alto
rendimiento
cuántico (55$-$67\%), excelente estabilidad de fotoluminiscencia térmica (hasta
200\textcelsius) y alta conversión de color, lo que hace que los fósforos
obtenidos
sean candidatos prometedores para aplicación en la fabricación de w-LED.\\

Debido al dopaje del granate anfitrión LuAG:Ce con iones \ce{Fe^{3+}} se
observó un
aumento en el tamaño de la red cristalina y de partícula, también permitió
generar propiedades magnéticas,
partiendo de una respuesta paramagnética a ferrimagnética, con un valor de
magnetización de saturación de $\sim$
10 emu/g a un campo aplicado relativamente bajo de $\sim$ 1500 Oe.

%El estudio presenta el proceso de sinterización del sistema
%\ce{Lu3Al_{5-x}Fe_{x}O_{12}}:\ce{Ce^{3+}} por método
%de reacción de estado sólido, el análisis realizado de las propiedades
%estructurales, ópticas, morfológicas, composicionales y magnéticas,
%mediante la caracterización usando técnicas de difracción de rayos X,
%refinamiento Rietveld, reflectancia difusa, absorción óptica,
%fotoluminiscencia, eficiencia Cuántica, microscopia electrónica de barrido
%(MEB) y magnetometría de muestra vibrante (VSM) Evaluar el efecto de la
%inclusión del catión Fe3+ sobre las propiedades
%estructurales, magnéticas y ópticas de los óxidos mixtos obtenidos.\\

%Se evaluó el efecto que tuvo la inclusión del ión \ce{Fe^{3+}} dentro de la
%estructura del granate LuAG sobre las propiedades
%estructurales, ópticas y magnéticas; empleando el sistema
%\ce{Lu3Al_{5-x}Fe_{x}O_{12}}:\ce{Ce^{3+}} con $0.0 \leq x \leq 4.5$.\\

%La investigación experimental que permitió conocer la influencia del
%\ce{Fe^{3+}} sobre las alteraciones en la propiedades del granate LuAG

%El resumen es una presentación abreviada y precisa (la NTC 1486 de 2008
%recomienda revisar la norma ISO 214 de 1976). Se debe usar una extensión
%máxima de 12 renglones. Se recomienda que este resumen sea analítico,
%es decir, que sea completo, con información cuantitativa y cualitativa,
%generalmente incluyendo los siguientes aspectos: objetivos, diseño, lugar y
%circunstancias, pacientes (u objetivo del estudio), intervención,
%mediciones y principales resultados, y conclusiones. Al final del resumen se
%deben usar palabras claves tomadas del texto (mínimo 3 y máximo 7
%palabras), las cuales permiten la recuperación de la información.\\

%\textbf{\small Palabras clave: (máximo 10 palabras, preferiblemente
%seleccionadas de las listas internacionales que permitan el indizado
%cruzado)}.\\

%A continuación se presentan algunos ejemplos de tesauros que se pueden
%consultar para asignar las palabras clave, seg\'{u}n el \'{a}rea
%temática:\\

%\textbf{Artes}: AAT: Art y Architecture Thesaurus.

%\textbf{Ciencias agropecuarias}: 1) Agrovoc: Multilingual Agricultural
%Thesaurus - F.A.O. y 2)GEMET: General Multilingual Environmental Thesaurus.

%\textbf{Ciencias sociales y humanas}: 1) Tesauro de la UNESCO y 2) Population
%Multilingual Thesaurus.

%\textbf{Ciencia y tecnología}: 1) Astronomy Thesaurus Index. 2) Life
%Sciences Thesaurus, 3) Subject Vocabulary, Chemical Abstracts Service y 4)
%InterWATER: Tesauro de IRC - Centro Internacional de Agua Potable y
%Saneamiento.

%\textbf{Tecnologías y ciencias médicas}: 1) MeSH: Medical Subject
%Headings (National Library of Medicine's USA) y 2) DECS: Descriptores en
%ciencias de la Salud (Biblioteca Regional de Medicina BIREME-OPS).

%\textbf{Multidisciplinarias}: 1) LEMB - Listas de Encabezamientos de Materia y
%2) LCSH- Library of Congress Subject Headings.\\

%También se pueden encontrar listas de temas y palabras claves, consultando
%las distintas bases de datos disponibles a través del Portal del Sistema
%Nacional de Bibliotecas\footnote{ver: www.sinab.unal.edu.co}, en la sección
%"Recursos bibliográficos" opción "Bases de datos".\\

%\textbf{\LARGE Abstract}\\\\
%Es el mismo resumen pero traducido al inglés. Se debe usar una
%extensión máxima de 12 renglones. Al final del Abstract se deben
%traducir las anteriores palabras claves tomadas del texto (mínimo 3 y
%máximo 7 palabras), llamadas keywords. Es posible incluir el resumen en
%otro idioma diferente al español o al inglés, si se considera como
%importante dentro del tema tratado en la investigación, por ejemplo: un
%trabajo dedicado a problemas lingüísticos del mandarín
%seguramente estar\'{\i}a mejor con un resumen en mandar\'{\i}n.\\[2.0cm]
%\textbf{\small Keywords: palabras clave en inglés(máximo 10 palabras,
%preferiblemente seleccionadas de las listas internacionales que permitan el
%indizado cruzado)}\\