\chapter{Experimental}

\section{Producción de los Materiales \ce{Lu3Al_{5-x}Fe_{x}O12}}

Los granates de \ce{Lu3Al5O12}:\ce{Ce^{3+}} dopados con \ce{Fe^{3+}} se
prepararon mediante un método de reacción en estado sólido, utilizando
precursores de alta pureza ($\geq$99,99\%) en polvo: \ce{Lu2O3}, \ce{Fe2O3},
\ce{Al(OH)3} y \ce{CeO2}. Los precursores fueron previamente calcinados a
800$^{\circ}$C 
durante 2 horas para eliminar impurezas volátiles, estas mezclas
estequiométricas según la composición granate \ce{Lu3Al_{5-x}Fe_{x}O12} con y x
= 0.0, 0.5, 1.0, 1.5, 2.0, 2.5, 3.0, 3.5, 4.0 y 0.45, se trituraron y prensaron
en pastillas bajo una presión axial de 2.5 MPa y se sinterizaron a
1200$^{\circ}$C  durante 20 horas \cite{Rivera2019}. Además, se utilizó carbón
activo para proporcionar una
atmósfera reductora.

\section{Caracterización}
\subsection{Difracción de Rayos X (DRX)}
Los patrones de difracción de DRX se registraron en un equipo PANalytical
X'Pert PRO-MPD con configuración Bragg-Brentano, usando radiación CuK$\alpha$
($\lambda$ = 1.5406 \r{A}), y un paso de escaneo de 0.02$^o$
en el rango
2$\theta$ 15– 75$^o$. Los datos registrados se utilizaron para análisis
semi-cuantitativo de las fases cristalinas, se utilizó el software X'Pert
HighScore versión 3.0 en conjunto con la base de datos PDF2-2004, para la
determinación de las fases constituyentes. El análisis estructural se realizó
mediante el método de refinamiento de Rietveld, utilizando los paquetes de
software GSAS14 y PCW15. La visualización en 3D de la celda unitaria resultante
se realizó con el paquete de software VESTA16.

\subsection{Reflectancia Difusa y Absorción Óptica}

Los espectros de reflectancia difusa y absorción óptica se obtuvieron usando un
espectrofotómetro Cary 5000 UV-Vis-NIR.

\subsection{Fotoluminiscencia y Eficiencia Cuántica}

Los espectros de fotoluminiscencia a temperatura ambiente y la
fotoluminiscencia dependiente de la temperatura se midieron con un
espectrofotómetro Hitachi F-7000, usando una lámpara Xe de 150 W como fuente de
luz. La eficiencia cuántica externa se midió utilizando una esfera de
integración recubierta de sulfato de bario como estándar de reflectancia en el
mismo espectrofotómetro.

\subsection{Magnetometría de Muestra Vibrante (VSM)}

Las medidas de magnetización fueron
realizadas en un magnetómetro de muestra vibrante VersaLab de Quantum Design, en
función de la temperatura y en función de un campo aplicado para un rango de
50 a 300 K y para campos magnéticos hasta 30 kOe. Se empleó el método ZFC-FC para
medidas de la magnetización en función de la temperatura.

\subsection{Microscopia Electrónica de Barrido (SEM)}

La morfología de las muestras se observó mediante microscopía electrónica de
barrido (SEM) utilizando el equipo JEOL JSM 6490-LV en 2k, 5k, 10k, 20k y 90k aumentos.
El análisis de espectrometría dispersiva de energía semicuantitativa (EDS) se
realizó utilizando un detector de elementos múltiples. Antes del análisis SEM,
las muestras se recubrieron con grafito para mejorar la conductividad.

