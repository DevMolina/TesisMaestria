\chapter{Conclusiones y recomendaciones}
\section{Conclusiones}

Se produjeron nuevos granates basados en el sistema
\ce{Lu3Al_{5-X}Fe_{x}O12}:\ce{Ce^{3+}} mediante el método de reacción en estado
sólido a 1200 °C durante 20 h. El dopaje \ce{Fe^{3+}} mejoró la pureza de los
materiales, obteniendo fase pura para valores de $x \geq  2.5$. Se observó una
expansión de la celda unidad y modificaciones en la energía de las absorciones
vibracionales al aumentar la concentración de \ce{Fe^{3+}}, lo que confirma la
adecuada
inserción del \ce{Fe^{3+}} en la estructura del huésped. El análisis óptico
indicó que
el dopaje con \ce{Fe^{3+}} condujo a una fuerte reducción de la banda prohibida
y la
intensificación de la energía del campo cristalino, que eran atribuibles a la
presencia de orbitales moleculares. Además, la fotoluminiscencia de emisión se
ajustó de verde a naranja debido a las modificaciones en los niveles de energía
de \ce{Ce^{3+}}. Los materiales obtenidos exhibieron propiedades ópticas, como
alta
conversión de color, estabilidad térmica, rendimiento cuántico y pureza de
color, que permiten sus posibles aplicaciones en la producción de w-LED.\\

Los resultados de caracterización morfológica realizados por medio de
microscopia electrónica de barrido MEB y el software ImageJ, mostraron que el
tamaño de partículas con forma irregular aumenta de manera proporcional a la
inclusión del \ce{Fe^{3+}} dentro de la estructura del granate, que se
relaciona
directamente con el aumento del tamaño del cristalito identificado mediante
el refinamiento Rietveld, obteniendo materiales con tamaños entre 100 y 350
nm.\\

La inclusión de \ce{Fe} dentro de la estructura del granate LuAG:Ce ajusto la
respuesta
magnética de los materiales, pasando por paramagnética, ferromagnética blanda y
ferromagnética.
Se observó que a mayor tamaño de partícula se tiene un mayor valor de
magnetización de saturación,
logrando 10 emu/g a un campo de 1500 Oe. Para los materiales con respuesta
ferrimagnética
($x \geq 3.5$) se evidencia una disminución de los valores de Mr y Hc a medida
que
aumenta la temperatura. Se observa un aumento significativo de Hc a medida que
la
temperatura disminuye, lo que indica que el material se vuelve magnéticamente
más
duro a menor temperatura.

%\section{Recomendaciones}
%Se presentan como una serie de aspectos que se podr\'{\i}an realizar en un
%futuro para emprender investigaciones similares o fortalecer la
%investigaci\'{o}n realizada. Deben contemplar las perspectivas de la
%investigaci\'{o}n, las cuales son sugerencias, proyecciones o alternativas que
%se presentan para modificar, cambiar o incidir sobre una situaci\'{o}n
%espec\'{\i}fica o una problem\'{a}tica encontrada. Pueden presentarse como un
%texto con caracter\'{\i}sticas argumentativas, resultado de una reflexi\'{o}n
%acerca de la tesis o trabajo de investigaci\'{o}n.\\